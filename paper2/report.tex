\documentclass[sigconf]{acmart}

\usepackage{graphicx}
\usepackage{hyperref}
\usepackage{todonotes}

\usepackage{endfloat}
\renewcommand{\efloatseparator}{\mbox{}} % no new page between figures

\usepackage{booktabs} % For formal tables

\settopmatter{printacmref=false} % Removes citation information below abstract
\renewcommand\footnotetextcopyrightpermission[1]{} % removes footnote with conference information in first column
\pagestyle{plain} % removes running headers

\newcommand{\TODO}[1]{\todo[inline]{#1}}

\begin{document}
\title{Big Data Analysis for Computer Network Defense}

\author{Jordan Simmons}
\affiliation{%
  \institution{Indiana University}
  \streetaddress{Smith Research Center}
  \city{Bloomington} 
  \state{IN} 
  \postcode{47408}
  \country{USA}}
\email{jomsimm@iu.edu}


% The default list of authors is too long for headers}
\renewcommand{\shortauthors}{G. v. Laszewski}


\begin{abstract}
Computer security threats and attacks are constantly evolving. Everyday, hackers are creating new techniques to bypass
network security for the purpose of malicious attacks. To keep up with the changing intrusion technologies, the technologies that defend these attacks need to constantly evolve also. Modern day technologies use deep learning techniques to monitor network activity, and detect malicious code. We will provide an overview of network security and modern technologies being used to protect computer systems and networks.
\end{abstract}

\keywords{i523,HID336, Computer Network Security, Big Data Analysis, Deep Learning, Intrusion Detection Systems, }


\maketitle

\section{Introduction}

Everyday a different computer network is being breached with the intent to cause harm to the system or to steal valuable data. Computer hackers are constantly creating new ways to evade network security and create malicious code that can not be detected by security systems. As malicious technologies continue to advance, the technologies that defend against these technologies need to adapt with these advances. The problem with computer network defence is that the technologies used to breach systems constantly change. Once a solution is created to defend a technology, a new malicious technology could be created the next day. Today many security specialist are using deep learning technologies to monitor network intrusions, and detect malicious code. In order to better understand computer network defense, an overview of modern attacks, network data collection processes, and the technologies used to analyze network data is provided. 


\section{Data Collection}
\subsection{Network Intrusion Data Collection}
\subsection{Malware Data Collection}


\section{Deep Learning for Network Intrusions}




\section{Deep Learning on Malware}



\section{Conclusion}




\begin{acks}

  The authors would like to thank Dr. Gregor von Laszewski for his
  support and suggestions to write this paper.

\end{acks}

\bibliographystyle{ACM-Reference-Format}
\bibliography{report} 

\end{document}
